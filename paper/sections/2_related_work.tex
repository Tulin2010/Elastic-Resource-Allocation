\section{Related work}
\label{sec:related-work}
Within Fog Computing, there exist a wide range of approaches to resource management including:
application placement, resource scheduling, task offloading, load balancing, resource allocation,
and resource provisioning~\cite{ghobaei2019resource}. As this work bridges a range of these topics, an outline of
related works from several approaches have been explained.

Auctions are a popular method for both dealing with self-interested users and in determining the value when allocating
multiple limited resources. \\
Due to the large corpus of research, Kumar et al. provides a systematic study of double auction mechanisms that are
suitable for a range of distributed systems like Grid computing, Cloud computing and Inter-Cloud
systems~\cite{KUMAR2017234}. The work reviewed 21 different proposed auction mechanisms over a range of important auction
properties like Economic Efficiency, Incentive Compatibility, and Budget-Balance. In a majority of the proposed auction
mechanisms, truthfulness was only considered for the user, thus a Truthful Multi-Unit Double auction mechanism was
presented as such that both users and servers should act truthfully. \\
Edge-MAP~\cite{tasiopoulos2018edge} provides a client-to-cloud model for tasks with extremely short deadlines
(10-50ms). Through using a Vickrey-English-Dutch (VED) auction, the system arrives at the unique minimum competitive
equilibrium price. Because of this, the system is highly scalable and adaptable to dynamically changing network
topologies. \\
An alternative market-based framework by Nguyen \emph{et al.}, allows for resources to be efficiently allocated by edge nodes
that are geographically distributed~\cite{8373684}. To maximise the use of node resource, the framework finds the market
equilibrium through optimally allocating resources bundles to services given each task's budget. The market's
equilibrium is found using the Eisenberg-Gale convex program which allows services to maximise their revenue. They
additionally proved a novel convex optimisation problem that achieves market equilibrium when services instead maximise
their net profit (revenue minus cost).

An advantage of fog nodes is that due to the proximity, nodes can find other nodes in its vicinity to assist in a
task~\cite{8839780}. Using this ability, resource allocation mechanisms have been proposed to allow fog nodes to offload
multiple tasks with delay guarantees. This is helpful due to the limited resources of nodes, this allows a node to
offload a task to either other fog nodes or remote cloud centers for further computational help. However this causes an
issue in deciding where to offload and how much partial task data to be offloaded under delay guarantee. By formulating
the problem as a Quadratically Constraint Quadratic Programming allows for a solution to be found.

Quality of service is an alternative metric for measuring the success of a system compared to social welfare used in a
majority of mechanisms. Chen et al. propose a resource-efficient computation offloading mechanism for users, where
communication and computation are joined by the node operation when allocating resources~\cite{8379445}. Using a task
graph model, the resource demands of a task are calculated, which are used to compute the optimal communication and
computation resource demand profile for a user in order to minimise a tasks resource occupancy. As the problem is a
NP-Hard problem, through an efficient approximation algorithm, a truthful pricing scheme is used to calculate the
critical value for the task to prevent users from misreporting task valuation.

As Fog Computing often requiring the placement of fog nodes within the system for both localised fog networks and
geographically distance systems due to the short deadline constraints of tasks. Work by Farhadi \emph{et al.}~\cite{vaji_infocom}, considers
the placement of code/data needed to run specific tasks over time where the demands change over time while also
considering the operational costs and system stability~\cite{vaji_infocom}. Using a proposed approximation algorithm,
they achieved 90\% of the optimal social welfare by converting the problem to a set function optimisation problem. By
doing this, this allows the algorithm to run in polynomial time complexity. \\
Alternative work aimed to efficiently distribute data centers and connections aims to maximise social
welfare instead of the number of tasks completed~\cite{Bi2019}. A truthful online mechanism was proposed that was
incentive compatible and individually rational, to allow tasks to arrive over time by solving an integer programming
optimisation problem.

All of the approaches explained above for task pricing and resource allocation in Cloud Computing use a form of fixed
resource allocation, where each user requests a fixed amount of resources for a task from a server. This is often
achieved through offering users a range of VM of different specifications for tasks to use. However, this mechanism,
as previously explained, provides no control for the server over the amount of resources allocated to a task, only
allowing the server to determine a task's price.

As our problem is related to multidimensional knapsack problems where there is a large body of
works~\cite{knapsacks, numbers}. Very little work has been done to allow for flexibility of item weights with linear
constraint that must be fulfilled for the item to be allocated~\cite{Nip2017}. The work provides a pseudo-polynomial
time complexity algorithm for solving this problem to maximize the values in the knapsack. While our optimisation
problem case is similar to this work, it differs due to the constraints with our using non-linear constraints making
it unusable.
