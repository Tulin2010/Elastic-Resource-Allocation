\section{Related work}
\label{sec:related-work}
There is a considerable amount of research in the area of resource allocation and pricing in cloud computing, some of
which use auction mechanisms to deal with competition~\cite{KUMAR2017234,Zhang2017,Du2019,Bi2019}.
However, these approaches assume that users request a fixed amount of resources system resources and processing rates,
with the cloud provider having no control over the speeds, only the servers that the task was allocated to. In our
work, tasks' owners report deadlines and overall data and computation requirements, allowing the edge cloud server to
distribute its resources more efficiently based on each task's requirements.

Other closely related work on resource allocation in edge clouds~\cite{vaji_infocom} considers both the placement of
code/data needed to run a specific task, as well as the scheduling of tasks to different edge clouds. The goal there is
to maximize the expected rate of successfully accomplished tasks over time. Our work is different both in the setup and
the objective function. Our objective is to maximize the value over all tasks. In terms of the setup, they assume that
data/code can be shared and they do not consider the elasticity of resources.

Our problem is related to multidimensional knapsack problems. In particular~\cite{Nip2017}, consider flexibility in
the allocation, with linear constraints that are used for elastic weights. The paper provides a pseudo-polynomial time
complexity algorithm for solving this problem to maximize the values in the knapsack. Our optimisation problem case is
similar to their, but differ due to constraints with our using non-linear not linear constraints.

A majority of approaches taken for task pricing and resource allocation in Cloud Computing uses a fixed resource
allocation mechanism, such that each user requests a fixed amount of resources for a task from a server. However this
mechanism, as previously explained, provides no control for the server over the quantity of resource allocated to a task,
only determining the task's price. As a result, a majority of approaches don't consider the server's management of
resource allocation. Thus research has focused on designing efficient and incentive compatible auction mechanisms.

Work by~\cite{KUMAR2017234} provides a systematic study of double auction mechanisms that are suitable for a range
of distributed systems like Grid computing, Cloud computing, Inter-Cloud systems. The work reviewed 21 different
proposed auction mechanisms over a range of important auction properties like Economic Efficiency,
Incentive Compatibility and Budget-Balance. In a majority of the proposed auction mechanisms, truthfulness was only
considered for the user, thus a Truthful Multi-Unit Double auction mechanism was presented as such that both users and
server should act truthfully.

Some approaches have been taken to increase flexibility within Fog Cloud Computing~\citep{Bi2019} through efficient
distribution of data centers and connections to maximise social welfare. A truthful online mechanism was
proposed that was incentive compatible and individually rational, to allow tasks to arrive over time by solving an
integer programming optimisation problem. Similar research in~\cite{vaji_infocom}, considers the placement of code/data
needed to run specific tasks over time where the demands change over time while also considering the operational costs
and system stability. An approximation algorithm achieved 90\% of the optimal social welfare by converting the problem
to a set function optimisation problem.