%! suppress = LabelConvention
\section*{Appendices}

\subsection*{Appendix A: Example problem case task and server attributes}
Within Subsection~\ref{subsec:example-problem-case} and~\ref{subsec:greedy-algorithm}, this example problem case is used
to demonstrate the effectiveness of elastic resource allocation compared to fixed resource allocation.
\begin{table}[h]
    \begin{minipage}{1.8in}
        \caption{Table of server attributes}
        \begin{tabular}{|c|c|c|c|}
            \hline
            Name     & $S_i$ & $W_i$ & $R_i$ \\ \hline
            Server 1 & 500   & 95    & 220   \\ \hline
            Server 2 & 500   & 95    & 210   \\ \hline
            Server 3 & 500   & 90    & 250   \\ \hline
        \end{tabular}
    \end{minipage}
    \begin{minipage}{3.5in}
        \caption{Table of task attributes. The columns ($s^{'}_j$, $w^{'}_j$, $r^{'}_j$) are fixed speeds which are not
        considered by the flexible resource allocation.}
        \begin{tabular}{|c|c|c|c|c|c|c|c|c|}
            \hline
            Name    & $v_j$ & $s_j$ & $w_j$ & $r_j$ & $d_j$ & $s^{'}_j$ & $w^{'}_j$ & $r^{'}_j$ \\ \hline
            Task 1  & 100   & 100   & 100   & 50    & 10    & 24        & 30        & 20        \\ \hline
            Task 2  & 90    & 75    & 125   & 40    & 10    & 23        & 27        & 19        \\ \hline
            Task 3  & 110   & 125   & 110   & 45    & 10    & 35        & 28        & 18        \\ \hline
            Task 4  & 75    & 100   & 75    & 60    & 10    & 25        & 25        & 20        \\ \hline
            Task 5  & 125   & 85    & 90    & 55    & 10    & 21        & 27        & 21        \\ \hline
            Task 6  & 100   & 75    & 120   & 40    & 10    & 20        & 29        & 19        \\ \hline
            Task 7  & 80    & 125   & 100   & 50    & 10    & 33        & 28        & 19        \\ \hline
            Task 8  & 110   & 115   & 75    & 55    & 10    & 30        & 22        & 20        \\ \hline
            Task 9  & 120   & 100   & 110   & 60    & 10    & 30        & 28        & 22        \\ \hline
            Task 10 & 90    & 90    & 120   & 40    & 10    & 27        & 27        & 18        \\ \hline
            Task 11 & 100   & 110   & 90    & 45    & 10    & 25        & 27        & 20        \\ \hline
            Task 12 & 100   & 100   & 80    & 55    & 10    & 24        & 24        & 22        \\ \hline
        \end{tabular}
    \end{minipage}
    \label{tab:example-tasks-server-properties}
\end{table}

\subsection*{Appendix B: Synthetic Model}
For the evaluation of the work in section~\ref{sec:empirical-results}, we used the following models. The left model
for the majority of the analyse with static one-shot cases and the right model for the analysing the online and batch
resource allocation methods.
\begin{table}[h]
    \begin{minipage}{2.8in}
        \begin{tabular}{|c|c|c|}
            \hline
            Attribute Name & Mean & Standard Deviation \\ \hline
            $v_j$          & 50   & 20                 \\ \hline
            $s_j$          & 100  & 15                 \\ \hline
            $w_j$          & 100  & 15                 \\ \hline
            $r_j$          & 50   & 10                 \\ \hline
            $d_j$          & 10   & 2                  \\ \hline
            $S_i$          & 450  & 50                 \\ \hline
            $W_i$          & 70   & 25                 \\ \hline
            $R_i$          & 290  & 45                 \\ \hline
        \end{tabular}
        \caption{Table of task and server attribute mean and standard deviations
        (used in section~\ref{sec:empirical-results})}
    \end{minipage}
    \begin{minipage}{2.8in}
        \begin{tabular}{|c|c|c|}
            \hline
            Attribute Name & Mean & Standard Deviation \\ \hline
            $v_j$          & 50   & 20                 \\ \hline
            $s_j$          & 100  & 15                 \\ \hline
            $w_j$          & 100  & 15                 \\ \hline
            $r_j$          & 50   & 10                 \\ \hline
            $d_j$          & 10   & 2                  \\ \hline
            $S_i$          & 400  & 40                 \\ \hline
            $W_i$          & 40   & 10                 \\ \hline
            $R_i$          & 150  & 20                 \\ \hline
        \end{tabular}
        \caption{Table of task and server attribute mean and standard deviations for the online work
        (used in subsection~\ref{subsec:comparison-between-online-and-batched-resource-allocation})}
    \end{minipage}
    \label{tab:synthetic-models}
\end{table}

\subsection*{Appendix C: Server Relaxed O
ptimisation Problem}
As an upper bound, a server relaxed version of the optimisation problem is used in
Subsection~\ref{subsec:evaluation-of-the-greedy-algorithm}. This is the optimisation problem solved by the branch and
bound solution.
\begin{align}
    \max & \sum_{\forall j \in J} v_j x_j \label{eq:relaxed-objective} \\
    \mbox{s.t.} \nonumber \\
    & \sum_{\forall j \in J} s_j x_j \leq S_i, \label{eq:relaxed-server-storage-constraint} \\
    & \sum_{\forall j \in J} w^{'}_j x_j \leq W_i,  \label{eq:relaxed-server-computation-constraint} \\
    & \sum_{\forall j \in J} (r^{'}_j + s^{'}_j) \cdot x_j \leq R_i \label{eq:relaxed-server-bandwidth-constraint} \\
    & \frac{s_j}{s^{'}_j} + \frac{w_j}{w^{'}_j} + \frac{r_j}{r^{'}_j} \leq d_j, &~ \forall{j \in J} \label{eq:relaxed-task-deadline} \\
    & 0 < s^{'}_j, &~ \forall{j \in J} \label{eq:relaxed-loading-speeds} \\
    & 0 < w^{'}_j, &~ \forall{j \in J} \label{eq:relaxed-compute-speeds} \\
    & 0 < r^{'}_j, &~ \forall{j \in J} \label{eq:relaxed-sending-speeds} \\
    & x_j \in \{0, 1\}, &~ \forall{j \in J} \label{eq:relaxed-task-allocation}
\end{align}